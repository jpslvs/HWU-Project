\documentclass[multicol]{elsarticle}
\usepackage[utf8]{inputenc}
\usepackage{lineno,hyperref}
\usepackage{multicol}
\usepackage{geometry}

\begin{document}
\begin{frontmatter} 
\title{New Ecological Data to Inform Conservation of Endangered Wetland Obligate \textit{Lilaeopsis schaffneriana} subsp. \textit{recurva}}
\author{J.P. Solves, K. Barron, L. Howard, B. Scott, J. Stromberg}
\address{427 E. Tyler Mall \\School of Life Sciences, Arizona State University, Tempe, Arizona}
\date{December 2018}

\begin{abstract}
This template helps you to create a properly formatted \LaTeX\ manuscript.
\end{abstract}

\begin{keyword}
\text{template}

\end{keyword}

\end{frontmatter}

\begin{multicols}{2}

\section{Introduction}
The genus Lilaeopsis consists of low growing, perennial, rhizomatous herbs, and is a member of the carrot family (Apiaceae). The genus is monophyletic with species that mainly exhibit a semi-aquatic to aquatic lifestyle (Affolter, 1985); Bone et al. 2011). Lilaeopsis species follow both amphitropic and amphiantarctic patterns of global distribution (Affolter 1985; Table 1). They are well-developed along temperate zones of the East and Northwest coasts of North America, South America, as well as throughout Australasia (Affolter, 1985). While the worldwide distribution of this genus suggests that it can persist in a variety of geographic settings, a majority of species remain restricted to a narrow range of habitat conditions (Affolter, 1985). The unifying factor among most occurrences is the cyclical presence of water in either freshwater or brackish environments (Affolter, 1985). Water cycles can either vary on a daily, or a seasonal basis depending on the habitat (intertidal zones, cienegas, temperate wetlands, etc), but all Lilaeopsis species tend to grow in partially or entirely submerged conditions where the soil is soggy or saturated (Affolter, 1985). The availability and flow of water has been observed to be a strong determinant in the overall survival and persistence of Lilaeopsis species (Stevenson, 1947, Radke et al, 2017). Greenhouse and field observations of Lilaeopsis suggest that this genus does not fare well in direct competition with other species and may rely on an intermediate degree of disturbance (scouring from floods, drought, etc) in order to persist among more aggressive herbaceous species (Affolter, 1985, Stevenson 1947, Titus & Titus, 2008). General observations also note that species of Lilaeopsis rely on their survival ability rather than competitive ability in order to persist in a locality (Stevenson, 1947). In order to survive, species of this genus have evolved an array of adaptations and behaviors such as forming persistent seed banks, colonizing via rhizomatous growth, and producing hydrochorus seeds. These traits have rendered Lilaeopsis species capable of enduring a requisite degree of disturbance. However, the parochial habitat requirements of this genus also render it sensitive to major environmental shifts, and one of the taxa that are considered rare or endangered as a result habitat loss (NatureServe, 2009). Lilaeopsis schaffneriana subsp. recurva AUTHOR, the Huachuca Water Umbel (hereafter HWU), is a cryptic subspecies with the typical morphology representative of the genus. This plant is an aquatic to semi-aquatic, perennial plant which occurs in freshwater 5–15 cm deep with clay or silt substrates containing organic content (United States Fish and Wildlife Service, 2014). It produces light yellow-green, thin, hollow, segmented leaves 1.0 to 3.0 mm wide, which grow in clusters generally about 4 to 8 cm long but can of grow up to 22.5 cm. The HWU reproduces both asexually through rhizomes, and sexually through small umbeliferous flowers. Flowers are white, less than 2.0 mm wide, and comprised of between 3 and 10 petals. The flowers produce oblong fruits that measure about 1.5 by 2.0 mm.  Seeds have spongy ribs that result in buoyancy, which allow for easy dispersal in water during flood events or in stream currents (Arizona Game and Fish Department: Heritage Data Management System YEAR?). This plant is also highly endemic to desert wetlands or “cienegas” of southeastern Arizona and northern Sonora, Mexico. These habitats are often characterized by mid-elevation perennial springs or headwater streams with saline, saturated soils (Hendrickson & Minckley, 1984). Because of their stable hydrology these habitats serve as important refugia for arid-sensitive species like the HWU. By the same token because of their unique hydrology and geology these wetlands in the American Southwest high rates of endemism, and support a large diversity of habitats (Hendrickson and Minckley, 1985). Nevertheless, their conservation remains largely deprioritized relative to other ecosystems with higher measures of alpha diversity (Cavieres et al. 2002). (GEOLOGY of HWU)Wetlands are defined by Cowardin and Golet as “lands transitional between terrestrial and aquatic systems where the water table is usually at or near the surface or the land is covered by shallow water” (Cowardin & Golet 1995). The hydro period of a wetland plays a key role in determining the composition of its plant communities, but so too does the xeroperiod. Cienegas in arid regions are characterized by prolonged xeroperiods interspersed by sustained flooding (Deil, 2005). Among other factors causing declines in HWU populations, climate change models predict increasing aridity throughout the American Southwest, with potential increase in the duration of dry soils during sustained drought (Melilo et al. 2014).  Effects of the duration of dry periods on plant species composition is receiving increased attention (Merlin et al. 2015) but there is much to be learned about species-specific tolerances to droughts of differing length. Many of the plants that grow in ephemeral wetlands are annuals or short-lived perennials that survive the dry period as dormant seeds (Stromberg et al. 2009; Brock 2011).  While rhizomes of the HWU have been anecdotally observed to survive short-term stress (Titus and Titus 2008b, c, Malcom et al. 2017), past studies on the drought tolerance of this species have shown that survivability is largely influenced by leaf density, colony history, and distance from water (). If a colony has been established in a locality for a certain time and has produced significant enough vegetative mass, the desiccation suffered from drought stress is manageable (Malcom et al. 2017). However, the longevity of its rhizomes during drought periods is still unknown. Preservation of the colony underground as a rhizomatous network during periods of drought could be a potential strategy for the HWU. Rhizome longevity would also provide a possible survival timeline for in situ conservation efforts during notoriously dry years. Along with forming rhizomes, persistent soil seed banks are an alternative survival strategy which enable plant populations to survive periodic drought, ecosystem disturbance, or otherwise unsuitable growing conditions. These seed banks form when seeds become incorporated into the soil and remain viable, but perhaps dormant (Warr et al. 1993). In its broadest definition the seed bank has been defined as ‘all the detached viable seeds (including fruits) of a species at a specific time, including seeds present both above and below the soil surface’ (Thompson and Grime, 1979), and has even been expanded to include viable seeds still attached to the parent plant (Cavers, 1995 & Goodson J.M. et al., 2001). The HWU has been observed to form persistent seed banks (Affolter 1985; Titus and Titus 2008a), and empirical studies suggest its seeds may remain viable for up to ten years (unpublished data in Titus and Titus 2008b), but this too remains to be quantified. In many ecosystems, including flood-scoured riparian habitats, the persistent soil seed banks tend to be dominated by short-lived plants including annual species (Capon and Brock 2006). Nevertheless, perennial herbaceous plants, which often are the dominant at wetland sites with low-intensity flooding cycles, have also be observed to form persistent soil seed banks (van der Valk 2013). While many plant species may form seed banks, the longevity of the seeds in the soil will widely vary among species by several orders of magnitude, owing to inherent differences in factors such as presence or absence of endosperm (Stromberg et al. 2009; Merritt et al. 2014). Within species, persistence of dispersed seeds is influenced by a suite of abiotic factors including temperature, humidity, soil pH, and soil C:N ratio (Pakeman et al. 2012; Long et al. 2015). Within families, geographical origin also has influence. Taxa that evolved in hot and dry climates tend to have greater seed longevity than those from cool and wet climates (Long et al. 2015). Organisms including granivores and pathogens similarly play determinant roles in deciding seed fates. Seeds that survive in the soil for less than one year are classified as transient while those that survive for longer periods are considered persistent (Csontos & Tamas 2003). Certain seed bank studies have shown that viable seeds of some wetland species can persist in the soil for decades after these areas have been dewatered or drained (Boudell & Stromberg 2008; van der Valk 2013). These results suggest that wetland plant communities could be restored if stream flows were reinstated and/or aquifers recharged. This restoration approach is time sensitive, however, given the varying longevities of wetland plant seeds (Merritt et al. 2014). Still, more definitive information on seed longevity is needed for this endangered species. Knowledge of seed traits is important for managing ex situ seed banks and for understanding extinction risk for species in the wild (Freville et al. 2007; Long et al. 2015). Current conservation efforts to maintain HWU populations in the wild has involved several transplants across its historic range to re-establish seed banks in these localities. Several of these transplants have already been reintroduced into historically extirpated field sites and protected urban sites. Conservation efforts have been successful, with respect to survivorship of transplants and short-term development of seed banks (Titus and Titus 2008b, c), but general ecological monitoring of this species still remains largely absent. This is possibly due to an array of factors which impact long term ecological monitoring conditions. The sites of extant populations, and certain field transplants are moderately difficult to access. A semi-aquatic life cycle, low growth form, and cryptic inflorescence also render it difficult to identify and track in the field. Nevertheless, monitoring the development of these transplants has been largely lacking, and the conservation of this plant would greatly benefit from long-term observational data. One study performed in 2010 involved monitoring a transplanted colony currently established in the Conservation Center at the Phoenix Zoo. Stuart Wells, the Conservation Director at the time, studied the reproductive response of the HWU to different water regimes (Wells & Morrow, unpublished manuscript). In the study, two habitat types were observed: aquatic (presence of standing water), and terrestrial (absence of standing water). It was found that in aquatic conditions, the HWU would flower and produce seeds, whereas when growing in terrestrial conditions, lacking standing water, the HWU produced neither fruit nor flower, but instead spread via clonal rhizomatous growth. This behavior slightly defied expectations. In related wetland taxa, the primary response to unfavorable habitat conditions (e.g., low water availability) tends to be sexual reproduction to increase genetic diversity and resilience, while in favorable conditions, plants tend to reproduce asexually to avoid expediting energy in costly reproductive units (Wells & Morrow, unpublished manuscript). It was suggested that since the HWU has adaptive features for seed dispersal via hydrochory, it is possible that the observed behavior is a result of the HWU taking advantage of ideal habitat conditions for seed dispersal. Nevertheless, low abundance of flowers and seeds in low-water conditions has the potential to be problematic for transplanted colonies impacted by increasing drought. If the HWU avoids fruiting in dry soil conditions, colonies in the wild which experience this stress will lose genetic diversity which would only exacerbate their survival conditions. Findings from a recent genetic analysis of the HWU populations confirm this problem. Reports from the Desert Botanical Garden show that “clonal growth is the primary mode of reproduction for the HWU, that genetic diversity within populations is very low, and that while genetic differences among populations and watersheds exist, they should be significantly considered when planning reintroductions” (Fehlberg, 2017). It is suggested that conservation efforts should consider preserving multiple, genetically distinct populations as well as local population connectivity, and suitable habitat for the establishment of new clones (Fehlberg, 2017).When considering past studies on the HWU, it becomes apparent that there exists a lack of cohesion between studied behaviors, habitat conditions, and conservation practices. The goal of this study is to attempt to fill gaps in knowledge on the ecology of the HWU via several experiments aimed at further elucidating survival and reproductive behaviors. The major objectives of this investigation are to address several aspects of the HWU life history. This investigation monitored for the establishment of a viable seed bank at extirpated and transplanted sites, measured the longevity of rhizomes during periods of drought, quantified the tentative longevity of seeds, and observed the impact of habitat type on reproduction strategy and timing. With these experiments, it will hopefully become possible to start addressing some of the more current problems facing HWU conservation. By providing insight into these traits, we hope to outline various new avenues of investigation as well as support ongoing conservation efforts to maintain viable populations of HWU in the wild. Also, by furthering our understanding of its ecology, we hope to highlight the importance of habitat conservation and restoration for the continued management of this endangered species.

\section{Methods}
\begin{subsection}{Seed Bank Establishment}
\,
Soil cores were collected from ten field sites across Arizona (Table 1-1). At five of these sites L. schaffneriana var. recurva had been historically observed but was currently extirpated, while the other five housed extant transplanted colonies of L.s. recurva. Transplant dates ranged from 3 to 13 years prior to this study. Extirpation dates were more difficult to determine and are not listed as a consequence. Soil cores were collected along a 10m line transect from the upper 5cm of soil. Each core was harvested using a metal cylinder at each of ten contiguous 1m2 plots along the stream or pond margin of each site. Two 100cm3 subsamples were also collected at each plot and combined in a plastic bag. Soil cores were transported in a cooler and kept refrigerated to avoid potential desiccation of seeds. At each plot abundance of associated common plant species, canopy cover, and distance from water source (if water present) were recorded. GPS coordinates were taken at the beginning and end of each 10m belt transect. If Lilaeopsis was present, the size of the population was estimated as ground cover percentage. Each study site was photographed for archival purposes (Figs.1-1 through 1-8). Soil cores for each site were grown out in a greenhouse using 10 x 20 planting trays containing six inserts (total of 240 inserts). Each core was split between two inserts, to increase surface area. Each insert contained a coffee filter (lining the bottom to prevent substrate loss), a 1cm layer of sterile sand substrate (to mimic the field substrate), and the soil core was placed on the top. The sand was prepped and autoclaved through a 10-hour sterilization cycle prior to use, and the collected soil layered on top of the sand remained refrigerated until use. Of the six inserts per tray, five contained soil cores and one contained only sand which served as a control for each individual tray. The trays were initially irrigated via a watering system consisting of distribution heads, spigots, and plastic tubes. Each head allocated water to several tubes, with a single spigot per tray. Each spigot head released approximately four gallons of water per hour. The irrigation system was used during first three weeks of the study but resulted in uneven watering levels; burst pipes occasionally disturbed the water pressure throughout the system. Consequently, manual watering of the trays was decided to be the preferred alternative. Every other day each tray was filled with 1000ml of water to maintain soil moisture moderately constant. The temperature in the greenhouse was approximately 28oC during daylight hours. The experiment was monitored for 12 weeks, through December 12, 2016.  Presence or absence of Lilaeopsis schaffneriana var. recurva per tray was recorded twice a week. Presence of other taxa was recorded on a weekly basis. To increase certainty of species identification of L.s. recurva, stem cross sections were examined under a dissecting scope. Trays were photographed to monitor overall species presence during the 1st, 4th, 8th, and 12th weeks of the study.
\end{subsection}

\begin{subsection}{Drought Tolerance}
\,
A hydroponic watering system was designed to create individual drought treatments for 12 L.s. recurva colonies. Treatments included five drought duration lengths (1-week, 2-week, 3-week, 4-week, and 6-weeks) and a continuously watered control (Appendix 2-1). There were two replicate tubs per treatment. Periods of drought were induced by artificially ceasing water flow and letting any remaining water evaporate. At the end of each drought treatment water flow was restored. Colonies used in the experiment were obtained from the Desert Botanical Garden. Colonies were placed in standard 10 x 20 planting trays with six equally divided inserts (Fig. 2-1). Transplants of equal size were transferred into twelve tubs with dimensions of 18” X 24”X 6” (Fig. 2-2).  Each tub contained rock wool, a horticultural growing substrate. The L.s. recurva were placed on the rock wool substrate and allowed to reproduce rhizomatously in order to fill the rock wool slabs. The rock wool was surrounded by a clay pebble growth medium to a depth of approximately 2”. Water was supplied to the tubs by a Ponics model PP8006 high lift pump which initially ran for six, fifteen-minute intervals spaced every two hours each day. The watering frequency was increased to 15 minutes every hour to better mimic stream flow patterns. Water was delivered through a branching network of ½” flexible tubing. Each tub was fitted with 4” of tubing capped with a restrictor to stabilize the pressure throughout the network. The tubs also had 1” diameter drainage holes topped with 1” risers and filter caps. The system is such that gravity returns the water to a 36- gallon reservoir tank via two PVC tracks fed by ½” flexible tubing attached to the drains in each tray. Aerators were included in the reservoir to increase available oxygen for L.s. recurva. Nutrients were supplied using a liquid nutrient solution from Floranova. The solution has a 7-4-10 ratio of N, P2O5, K2O, and micronutrients. Water was monitored and adjusted to maintain pH and appropriate levels of nutrients. 
\end{subsection}

\begin{subsection}{Seed Longevity}
\,The original protocol intended was to determine longevity as a result of artificially aging seeds (via high temperature and humidity) and then testing for viability, but this required a greater number of seeds than were available. Instead, a time series of seeds of known age was used. L.s. recurva seeds of varying age (from 1 to 15 years) were acquired from Dr. Titus of the State University of New York-Fredonia (SUNY) (Table 3-1). The seeds had been stored in paper packets. Some had been directly collected from field sites in Arizona and some had been sent from the Desert Botanical Garden (DBG) to Dr. Titus. The seeds had been stored in cool, dry conditions except during short periods when they were moved between locations. The contents of each seed packet (henceforth, population) was emptied into a petri dish, and examined under a dissecting microscope to check for damage and debris. Twenty percent of the total number of seeds from each population were placed into petri dishes on moist filter paper (Fig. 3-1). On July 22, 2016, ten petri dishes, with up to ten seeds per dish, were placed into a germination chamber. The chamber was set to 25oC and had alternating 12 hour cycles of UV lighting. Germination was recorded weekly during the 12-week period from July 22, 2016 to October 12, 2016. Water was added periodically to the petri dishes to maintain a moist substrate. At the end of the study, un-germinated seeds were sliced with razor blade and visually examined under a dissecting scope on to check for viability. 
\end{subsection} 

\begin{subsection}{Reproductive Response to Habitat}
\,
Location of transplanted HWU population were chosen for monitoring. Colonies at the Phoenix Zoo and Desert Botanical Garden were chosen and categorized as urban sites. Colonies occurring in ponds within Las Cienegas National Conservation Area were also chosen and categorized as field sites. A summary of site locality and associated information can be found in (FIG). Individual habitat types were then designated at each site. A terrestrial habitat was defined as a patch without observable water present at the base of the plants; this includes habitat with saturated moist soil as well as dry soil. An aquatic habitat was defined as habitat with observable water present at the base of the plants. Surveying was conducted using a 40 x 40 cm quadrat constructed of PVC pipe and nylon rope to create 16 individual sampling areas. Flags were used to mark specific location of the quadrats at all sites for replication. Eight quadrat locations were designated for the urban sites, and four quadrat locations were designated for the field sites. Each quadrat at each habitat type was carefully scored for buds, flowers, and fruit. Wells & Morrow found that the time from bud to flower was approximately 3.4 days, flower to fruit was approximately 3.8 days. As a result, in this study the presence and number of buds, flowers, and fruit at each habitat was recorded every 6-8 days in order to avoid double counts for 7 months from January to July 2018. This timeline followed the colonies from a dormant state during the winter months to peak flowering and fruiting time during pre-monsoon months. Leaf height and ground cover of HWU was measured and averaged within each quadrat. Environmental data at both urban and field sites was also be recorded. A mercury thermometer was used to measure the water temperature in aquatic habitat patches. In each quadrat (except with standing water) a soil moisture meter was used to determine moisture content. Daily weather data was monitored from the set GPS point. Long-term weather data was acquired from USGS weather stations as well as water pressure data loggers present within local wells near LCNCA. Daily precipitation data was gathered from nearby rainfall gauges and/or reputable weather sources. Time of day, time of sunrise, and sunset was recorded to establish photoperiod. Air temp as well as daily highs and lows was recorded. A general description of the populations and details about each occurrence, including approximate density of the cluster, and percentage ground cover was noted during each visit. Site elevation and any other distinguishing features not explicitly outlined in the data was also noted. Photographs of surrounding habitat and of quadrats was taken for documentation of population state. Most common associated species present at each HWU population was documented.
\end{subsection}

\section{Results}
\begin{subsection}{Seed Bank Establishment}
\,
Plants identified as L. schaffneriana were detected in soils of two of the five transplant sites- the Desert Botanical Garden pond edge and Las Cienega ponds. Specifically, it grew from the following soil cores: DBG 1-2, DBG 1-1, DBG 7-2, DBG 8-1, and LCC 5-2. Lilaeopsis were not detected in soils from the other three transplant sites- Phoenix Zoo stream, Finley Tank Springs, and Horsethief Springs- or from any of the five extirpated sites (FIG). There was a positive correlation between abundance (cover) of L. schaffneriana in the field and frequency of detection of seedlings in the soil seed bank trays. Notably, the two sites for which we detected emerging seedlings were the two sites that had the greatest cover of L. schaffneriana in the field (37 cover at DBG and 23 at Las Cienegas). Other plants that emerged from the collected soils included the wetland plant Eleocharis, various grasses, and many dicots (FIG). Four weeks into the study, many of the trays had developed a surface growth of fungus or algae. Control trays had no seedlings, indicating no contamination. At field sites where L. schaffneriana subsp. recurva was present, the dominant associated species were perennial wetland plants including Eleocharis, Schoenoplectus, and Juncus (FIG). At sites where it had been extirpated, dominant species included plants characteristic of seasonally dry riparian soils, such as Cynodon dactylon.  Tree cover ranged widely among sites from 77 at DBG to 0 at Las Cienegas. 
\end{subsection}

\begin{subsection}{Drought Tolerance}
\,
The experiment indicated that in a greenhouse setting L. schaffneriana var. recurva can survive two to three weeks of drought (FIG). Specifically, plants had: (1) 100 survivorship of one week of flow-cessation (which translated to one week of moist substrate). The substrate changed from saturated to moist during the first week of water-flow cessation, with the measured moisture level dropped by roughly 50. Plants showed signs of stress (browning of stems) but recovered (re-greened) after water flows were reinstated.  (2) 100 survivorship of two weeks of flow cessation (which translated to one week of moist substrate and one week of dry substrate). Plants appeared stressed the first week (browning of stems) and appeared dead in the 2nd week (with no green tissue apparent), but resprouted from rhizomes in the 3rd week and were fully green and vigorous by the 4th week. (3) 50 survivorship of three weeks of flow-cessation (which translated to one week of moist substrate and two weeks of dry). In one of the tubs receiving this treatment, the plants died and did not resprout after water flows were reinstated. In the other tub, plants initially appeared stressed and then appeared dead (leaves brown and dry). Plants ultimately resprouted from rhizomes in the 3rd week after water flows were reinstated (FIG). (4) No survivorship of four or more weeks of flow-cessation (three or more weeks of dry substrate).
\end{subsection}

\begin{subsection}{Seed Longevity}
\,
There was no apparent decline in germination rate with seed age. The oldest seeds, collected 15 years ago, had a high germination rate (80) (FIG). Germination rate varied widely among source populations, independent of age. For example, one of the 1-yr-old seed populations had a germination rate of 11 whereas another had 100 germination. Two populations of intermediate age (nine and 14 years) had no germination. Germination did appear to be slower for older populations. The first germination was observed four days after moistening of the seeds (FIG). The two young seed populations both germinated rapidly, with all germinants appearing during the first three weeks. Germination for older seeds (15-year-olds) continued sporadically through week ten.  Many ungerminated seeds became mold covered. Two of the seeds that had not germinated by the end of this experiment were considered viable based on the cut test, suggesting that they were physiologically dormant.  Mean weight of the seeds was determined to be 2.68 mg.
\end{subsection}

\begin{subsection}{Reproductive Response to Habitat}
\,
There was no difference observed in reproductive strategy from either habitat type or site type. Only aquatic habitat data was considered when comparing field and urban sites due to the absence of ‘terrestrial habitat’ at the field site locality. There was a difference observed in the abundance of reproductive units (buds, flowers, and fruit) in different habitat types and site types(FIG). Colonies observed in terrestrial habitat at both urban sites (DBG and ZOO) produced 212 more buds and 18 more fruit than colonies observed in aquatic habitats. With site locality, the field sites had a higher abundance of reproductive units compared to urban sites (FIG). Colonies observed in field sites produced 3951 more buds than colonies observed in urban sites. Overall percentage of floral buds which became fruit is calculated for both habitat and site types (FIG). This percentage indicates ratio of buds which matured into viable. There was a 0.08 difference in this ratio between aquatic and terrestrial habitat types (8.62 and 8.54 respectively), and a 30.38 difference between urban and field site types (8.62 and 39) (FIG).
\end{subsection}
\end{multicols}

\nocite{*}
\bibliographystyle{IEEEannot}
\bibliography{name}
\end{document}
